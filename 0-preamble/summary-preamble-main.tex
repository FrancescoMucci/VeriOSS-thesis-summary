%--------------------------------------------------------------
%               Preamble for thesis summary 
%--------------------------------------------------------------

%--------------------DOCUMENT-CLASS----------------------------

\documentclass[twoside = false, 12pt, a4paper]{scrbook}
% The scrbook class provides provides features useful when preparing a book for publication. This class provides the "book"-like element of the "koma-script", a collection of replacements for the "article", "report" and "book" classes with emphasis on typography and versatility.

% DEACTIVATAD
% Option "twoside": with this option the inner margin of one page is only half as wide as the corresponding outer margin; also the left and right margins are swapped on verso pages, so this option is ideal for double-sided printing. If the option is passed without a value, the value "true" is assumed, so two-sided printing is enabled. Deactivating the option with "twoside = false" leads to one-sided printing: the left and right margins are the same width.

%--------------------------------------------------------------

%--------------------SUBDIRECTORIES-SETTING--------------------

% The following code declares a list of subdirectories so you can insert TeX files ("\input", "\include"), images ("\includegraphics"), or code ("\lstinputlisting") without specifying the path to the inserted file (if contained in one of these subdirectories).

\makeatletter
\providecommand*{\input@path}{}
\g@addto@macro\input@path{{0-preamble/}{0-preamble/basic/}{0-preamble/math/}{0-preamble/keywords/}{1-front/}{2-main/}{other/}}
\makeatother

% Our list of subdirectories is hold by "\input@path": an internal command; these kind of commands use "@" in their name, are mainly intended for package or class writers and normal users are prevented from accidentally redefining them. To be able to define or redifine such commands, you need to use  the commands "\makeatletter" and "\makeatother"; the first one ensures that the "@" character is seen as a letter when searching for command names; the last one disable the possibility to act on internal commands to avoid subsequent problems with command interpretation.

% The "\input@path" macro is normally undefined in LaTeX, but can be defined by some used package: for example, "graphics" and "graphicx" internally store the path specified in "\graphicspath" in "\Ginput@path" and locally sets "\input@path" to "\Ginput@path". We can therefore use "\providecommand" to define "\input@path" in the the case that it is undefined and then "\g@addto@macro" to extends its definition.

%--------------------------------------------------------------

%--------------------PACKAGES----------------------------------

%--------------------------------------------------------------
%    		 Basic packages for thesis summary 
%--------------------------------------------------------------

%--------------------BASIC-PACKAGES----------------------------

\usepackage{geometry}
% The "geometry" package provides a flexible and easy interface to page dimensions. You can change the page layout with intuitive parameters.

\usepackage[italian]{babel}
% The "babel" package which enables LaTeX to typeset in many different languages.

\usepackage[utf8]{inputenc}
% The package "inputenc" translates various standard and other input encodings into a LaTeX internal language. It is used by LaTeX to correctly interpret the characters entered in the editor. "utf8" is the input encoding which allows you to write the signs of numerous alphabets in the editor directly from the keyboard, avoiding having to load the encoding suitable for the language of the document each time.

\usepackage{hyphenat}
% The "hyphenat" package can disable all hyphenation or enable hyphenation of non-alphabetics or monospaced fonts; this package also enable hyphenation within words that contain non-alphabetic characters and hyphenation of text typeset in monospaced fonts. In particular TeX does not hyphenate already hyphenated word, such as "electromagnetic-endioscopy"; in this case, the "\hyp" command can be used to allow automatic hyphenation of compound words: e.g. "electromagnetic{\hyp}endioscopy".

\usepackage{lipsum}
% The "lipsum" package gives you easy access to 150 paragraphs of the Lorem Ipsum dummy text.

%--------------BASIC-PACKAGES-LOADED-BY-CLASSICTHESIS----------

%\usepackage[T1]{fontenc} % WILL BE LOADED by "dia-classicthesis-ldpkg".
% The package "fontenc" allows the user to select font encodings, and for each encoding provides an interface to font-encoding-specific commands for each font. Its most powerful effect is to enable hyphenation to operate on texts containing any character in the font. "T1" is the output encoding for writing in Italian and in many other western languages.

%--------------------------------------------------------------

%--------------------SOURCES-FOR-COMMENTS----------------------

% The comments on LaTeX and its commands are based on the contents of https://latexref.xyz/, an unofficial reference manual for the LaTeX2e document preparation system.

% The comments on the classes, styles or packages (and their commands and options) come from the description provided on CTAN (https://www.ctan.org/) and from the official documentation of the different classes, styles or packages.

%--------------------------------------------------------------

%--------------------------------------------------------------
%         Math related packages for thesis summary
%--------------------------------------------------------------

%--------------------MATH-PACKAGES----------------------------

\usepackage[fleqn]{amsmath} 
% The "amsmath" package is a LaTeX package that provides various extensions for improving the information structure and printing of documents containing mathematical formulas. The "fleqn" option position equations at a fixed indent from the left margin rather than centered in the text column.

\usepackage{amssymb}
% The "amssymb" package provides additional math symbols, like arrows, operators, special characters, geometric figures.

\usepackage{mathtools}
% The "mathtools" package is an extension package to amsmath. There are two things on "mathtools" agenda: (1) correct various bugs/deficiencies in amsmath until these are fixed by the AMS and (2) provide useful tools for mathematical typesetting for example the ability to write over arrows.

\usepackage{stmaryrd}
% The "stmaryrd" packages provides a number of new symbols for theoretical computer science, including ones for derivation of functional programming, process algebra, domain theory, linear logic, multisets and many more. It also fixes some features with AMS symbols and adds obvious variants of others.

%--------------------SOURCES-FOR-COMMENTS----------------------

% The comments on LaTeX and its commands are based on the contents of https://latexref.xyz/, an unofficial reference manual for the LaTeX2e document preparation system.

% The comments on the classes, styles or packages (and their commands and options) come from the description provided on CTAN (https://www.ctan.org/) and from the official documentation of the different classes, styles or packages.

%--------------------------------------------------------------

%--------------------------------------------------------------
%    Basic packages for thesis summary to be loaded as last 
%--------------------------------------------------------------

%--------------------CLASSIC-THESIS-STYLE----------------------

% Classic Thesis Style: an easy-to-use template for Master’s or PhD thesis, Copyright (C) 2008 André Miede http://www.miede.de

\usepackage{dia-classicthesis-ldpkg}
\usepackage[beramono]{classicthesis}

% DEACTIVATED
% Option "drafting": prints the date and time at the bottom of each page, so you always know which version you are dealing with. Might come in handy not to give your Prof. that old draft.

% ACTIVATED
% Option "beramono": loads Bera Mono as typewriter font. Default setting is using the standard CM typewriter font.

%---------------------------------------------------------------

%--------------PACKAGES-TO-BE-LOADED-AFTER-HYPERREF-------------

\usepackage{ellipsis} % LOAD AFTER "hyperref".
% The "ellipsis" package fixes a problem in the way LaTeX handles ellipses ("\dots); LaTeX always puts a tiny bit more space after "\dots" in text mode than before it, which results in the ellipsis being off-center when used between two words.

%---------------------------------------------------------------

%--------------------SOURCES-FOR-COMMENTS-----------------------

% The comments on LaTeX and its commands are based on the contents of https://latexref.xyz/, an unofficial reference manual for the LaTeX2e document preparation system.

% The comments on the classes, styles or packages (and their commands and options) come from the description provided on CTAN (https://www.ctan.org/) and from the official documentation of the different classes, styles or packages.

%---------------------------------------------------------------

%--------------------------------------------------------------

%--------------------SETTINGS----------------------------------

%--------------------------------------------------------------
%               Settings for thesis summary 
%--------------------------------------------------------------

%--------------------GENERAL-SETTINGS--------------------------

\newlength{\abcd} % for ab..z string length calculation.
% "\newlength" is a LaTeX macro which defines a new length register, which holds a length as number and can be used for calculations.

\setlength{\parindent}{0cm} 
% To remove the leading indent of the new paragraph

%---------------------------------------------------------------

%--------------------PAGE-LAYOUT-SETTINGS-----------------------

% Setting the page layout using the "geometry" package.

% Summary page layout consistent with that of the thesis
%\geometry{
%	a4paper,
%	ignoremp,
%	bindingoffset = 1cm, 
%	textwidth     = 13.5cm,
%	textheight    = 21.5cm,
%	lmargin       = 3.5cm,
%	tmargin       = 4cm   
%}

% Summary page layout inconsistent with that of the thesis v1: no "bindingoffset", "rmargin" equal to "lmargin", reduced "tmargin" and "bmargin".
%\geometry{
%	a4paper,
%	ignoremp,
%	bindingoffset = 0cm, 
%	textwidth     = 13.5cm,
%	textheight    = 21.5cm,
%	lmargin       = 3.5cm,
%	rmargin       = 3.5cm,
%	tmargin       = 0.5cm,
%	bmargin       = 1.8cm  
%}

% Summary page layout inconsistent with that of the thesis v2: no "bindingoffset", "rmargin" equal to "lmargin" and both reduced, reduced "tmargin" and "bmargin".
\geometry{
	a4paper,
	ignoremp,
	bindingoffset = 0cm, 
	textwidth     = 13.5cm,
	textheight    = 21.5cm,
	lmargin       = 3.0cm,
	rmargin       = 3.0cm,
	tmargin       = 0.5cm,
	bmargin       = 2.0cm  
}

% The page layout in the "geometry" package contains a total body (printable area) and margins: the total body consists of a body (text area) with an optional header, footer and marginal notes; the margins are left, right, top and bottom (for twosided documents, horizontal margins should be called inner and outer).

% The "a4paper" specifies the paper size by name.

% The "ignoremp" option disregards the marginal notes in determining the horizontal margins (defaults to true).

% The "bindingoffset" option removes a specified space from the lefthand-side of the page for oneside printing or the inner-side for twoside printing. This is useful if pages are bound by a press binding.

% The "textwidth" option the width of the body.

% The "textheight" option sets the height of the body (including footnotes and figures, excluding running head and foot).

% The "lmargin" option sets the left margin (for oneside printing) or inner margin (for twoside printing) of total body: the distance between the left or inner edge of the paper and that of total body.

% The "rmargin" option sets the right margin (for oneside printing) or outer margin (for twoside printing) of total body: the distance between the right or outer edge of the paper and that of total body.

% The "tmargin" option sets the top margin of the page.

% The "bmargin" option sets the bottom margin of the page.

%--------------------------------------------------------------

%--------------------SOURCES-FOR-COMMENTS----------------------

% The comments on LaTeX and its commands are based on the contents of https://latexref.xyz/, an unofficial reference manual for the LaTeX2e document preparation system.

% The comments on the classes, styles or packages (and their commands and options) come from the description provided on CTAN (https://www.ctan.org/) and from the official documentation of the different classes, styles or packages.

%--------------------------------------------------------------

%--------------------------------------------------------------

%--------------------KEYWORDS----------------------------------

%--------------------------------------------------------------
%          Keyword helper commands for thesis summary
%--------------------------------------------------------------

%----------------------KEYWORDS---------------------------------

% The "\keyword{\<name>}{<text>}" command can be used to define a simple keyword: useful to never forget the "\xspace" at the end of the definition (without the "\xspace" every keywords inserted inside text should be followed or enclosed by curly brackets or else there would be no space added).

\newcommand{\keyword}[2]{%
	\newcommand{#1}{#2\xspace}%
}

% For testing
\newcommand{\simpleNewcommand}{newcommand}
\keyword{\keywordExample}{keyword}

%--------------------------------------------------------------

%----------------------EMPHASIZED-KEYWORDS---------------------

% The "\emphKeyword{\<name>}{<text>}" command can be used to define a keyword with emphasized text.

% Basic version.
%\newcommand{\emphKeyword}[2]{%
%	\newcommand{#1}{\emph{#2}\xspace}%
%}

% Improved version.
\newcommand{\emphKeyword}[2]{%
	\keyword{#1}{\emph{#2}}%
}

% For testing.
\emphKeyword{\emphKeywordExample}{emphasized-keyword}

%--------------------------------------------------------------

%----------------------TEXTTT-KEYWORDS-------------------------

% The "\ttKeyword{\<name>}{<text>}" command can be used to define a keyword with typewriter font (useful for code-related keywords).

% Basic version.
%\newcommand{\ttKeyword}[2]{%
%	\newcommand{#1}{\texttt{#2}\xspace}%
%}

% Improved version.
\newcommand{\ttKeyword}[2]{%
	\keyword{#1}{\texttt{#2}}%
}

% For testing.
\ttKeyword{\ttKeywordExample}{typewriter-keyword}

%--------------------------------------------------------------

%------------------HYPERLINKS-HELPER-COMMANDS------------------

% The "\myHref{<URL>}{<text>" command is a wrapper of "\href" from the "hyperref" package (made the text specified as the second argument a hyperlink to the URL specified as the first argument); in my version the text will be rendered using the typewriter font.

\newcommand{\myHref}[2]{%
	\href{#1}{\texttt{#2}}%
}

% The "\https{<URL-without-protocol>}" command can be used to reference a "secure" web page.
\newcommand{\https}[1]{%
	\myHref{https://#1}{#1}%
}

% The "\http{<URL-without-protocol>}" command can be used to reference a "not secure" web page.
\newcommand{\http}[1]{%
	\myHref{http://#1}{#1}%
}

% The "\mailto{<email>}" command can be used to reference an email address.
\newcommand{\mailto}[1]{%
	\myHref{mailto://#1}{#1}%
}

%--------------------------------------------------------------

%----------------------SECURE-WEBPAGE-KEYWORDS-----------------

% The "\webpage{\<name>}{<URL-without-protocol>}" command can be used to define a keyword for a "secure" webpage.

% Basic version.
%\newcommand{\webpage}[2]{%
%	\newcommand{#1}{\href{https://#2}{\texttt{#2}}\xspace}
%}

% Improved version.
\newcommand{\webpage}[2]{%
	\keyword{#1}{\https{#2}}%
}

% For testing.
\webpage{\myWebpage}{francescomucci.github.io}

%--------------------------------------------------------------

%----------------------NOT-SECURE-WEBPAGE-KEYWORDS-------------

% The "\notsecurewebpage{\<name>}{<URL-without-protocol>}" command can be used to define a keyword for a "not secure" webpage.

% Basic version.
%\newcommand{\notsecwebpage}[2]{%
%	\newcommand{#1}{\href{http://#2}{\texttt{#2}}\xspace}
%}

% Improved version.
\newcommand{\notsecwebpage}[2]{%
	\keyword{#1}{\http{#2}}%
}

% For testing.
\notsecwebpage{\notSecureWebpageExample}{icetcs.ru.is}

%--------------------------------------------------------------

%----------------------MAIL-KEYWORDS---------------------------

% The "\mail{\<name>}{<email>}" command can be used to define a keyword for an email address.

% Basic version.
%\newcommand{\mail}[2]{%
%	\newcommand{#1}{\href{mailto://#2}{\texttt{#2}}\xspace}
%}

% Improved version.
\newcommand{\mail}[2]{%
	\keyword{#1}{\mailto{#2}}%
}

% For testing.
\mail{\myTestMail}{francesco.mucci@edu.unifi.it}

%--------------------------------------------------------------

%--------------------SOURCES-FOR-COMMENTS----------------------

% The comments on LaTeX and its commands are based on the contents of https://latexref.xyz/, an unofficial reference manual for the LaTeX2e document preparation system.

% The comments on the classes, styles or packages (and their commands and options) come from the description provided on CTAN (https://www.ctan.org/) and from the official documentation of the different classes, styles or packages.

% The comments on TeX conditional commands are also based on "TeX by Topic" (2017) by Victor Eijkhout.

%---------------------------------------------------------------

%--------------------------------------------------------------
%           Keywords for thesis summary title
%--------------------------------------------------------------

\keyword{\myItalianTitle}{Progettazione di uno smart contract a supporto del protocollo di fair exchange di VeriOSS, una piattaforma bug bounty basata sulla blockchain}
\keyword{\myFaculty}{Scuola di Scienze Matematiche, Fisiche e Naturali}
\keyword{\myDegreeLevel}{Laurea Magistrale}
\keyword{\myDegree}{Informatica}
\keyword{\myYear}{Anno Accademico 2022-2023}

\keyword{\myName}{Francesco Mucci}
\mail{\myMail}{francesco.mucci@edu.unifi.it}

\keyword{\myProf}{Rosario Pugliese}
\mail{\myProfMail}{rosario.pugliese@unifi.it}

\keyword{\myCorelatore}{Gabriele Costa}
\mail{\myCorelatoreMail}{gabriele.costa@imtlucca.it}

\keyword{\myOtherCorelatore}{Letterio Galletta}
\mail{\myOtherCorelatoreMail}{letterio.galletta@imtlucca.it}

%--------------------------------------------------------------

%--------------------------------------------------------------

%--------------------SOURCES-FOR-COMMENTS----------------------

% The comments on LaTeX and its commands are based on the contents of https://latexref.xyz/, an unofficial reference manual for the LaTeX2e document preparation system.

% The comments on the classes, styles or packages (and their commands and options) come from the description provided on CTAN (https://www.ctan.org/) and from the official documentation of the different classes, styles or packages.

%--------------------------------------------------------------