%--------------------------------------------------------------
%    		 Basic packages for thesis summary 
%--------------------------------------------------------------

%--------------------BASIC-PACKAGES----------------------------

\usepackage{geometry}
% The "geometry" package provides a flexible and easy interface to page dimensions. You can change the page layout with intuitive parameters.

\usepackage[italian]{babel}
% The "babel" package which enables LaTeX to typeset in many different languages.

\usepackage[utf8]{inputenc}
% The package "inputenc" translates various standard and other input encodings into a LaTeX internal language. It is used by LaTeX to correctly interpret the characters entered in the editor. "utf8" is the input encoding which allows you to write the signs of numerous alphabets in the editor directly from the keyboard, avoiding having to load the encoding suitable for the language of the document each time.

\usepackage{hyphenat}
% The "hyphenat" package can disable all hyphenation or enable hyphenation of non-alphabetics or monospaced fonts; this package also enable hyphenation within words that contain non-alphabetic characters and hyphenation of text typeset in monospaced fonts. In particular TeX does not hyphenate already hyphenated word, such as "electromagnetic-endioscopy"; in this case, the "\hyp" command can be used to allow automatic hyphenation of compound words: e.g. "electromagnetic{\hyp}endioscopy".

\usepackage{lipsum}
% The "lipsum" package gives you easy access to 150 paragraphs of the Lorem Ipsum dummy text.

%--------------BASIC-PACKAGES-LOADED-BY-CLASSICTHESIS----------

%\usepackage[T1]{fontenc} % WILL BE LOADED by "dia-classicthesis-ldpkg".
% The package "fontenc" allows the user to select font encodings, and for each encoding provides an interface to font-encoding-specific commands for each font. Its most powerful effect is to enable hyphenation to operate on texts containing any character in the font. "T1" is the output encoding for writing in Italian and in many other western languages.

%--------------------------------------------------------------

%--------------------SOURCES-FOR-COMMENTS----------------------

% The comments on LaTeX and its commands are based on the contents of https://latexref.xyz/, an unofficial reference manual for the LaTeX2e document preparation system.

% The comments on the classes, styles or packages (and their commands and options) come from the description provided on CTAN (https://www.ctan.org/) and from the official documentation of the different classes, styles or packages.

%--------------------------------------------------------------