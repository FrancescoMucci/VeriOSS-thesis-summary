%--------------------------------------------------------------
%            Main text for UniFI thesis summary 
%--------------------------------------------------------------

\textbf{Contesto.} Identifichiamo il campo di studi che costituisce lo sfondo del nostro lavoro di tesi e introduciamo lo specifico problema che intendiamo affrontare; per entrambi, evidenzieremo la rilevanza e le motivazioni per cui meritano attenzione.

\medskip

\textbf{Obiettivi.} Delineiamo le domande di ricerca che guideranno il nostro studio, cioè le domande finalizzate ad identificare le lacune nelle conoscenze esistenti che cercheremo di colmare. Ad ognuna di queste domande sarà associato un obiettivo che la nostra tesi si prefigge di raggiungere; tali obiettivi indicano cosa abbiamo intenzione di dimostrare, sviluppare, testare o esplorare. Se le domande di ricerca non risultassero chiare senza prima aver introdotto i lavori precedenti su cui si fonda la nostra ricerca, potrebbe essere opportuno posticipare la presentazione delle suddette domande.

\medskip

\textbf{Metodologia.} Descriviamo l'approccio adottato per risolvere il problema affrontato, fornendo una sintesi della soluzione proposta. Dovremo menzionare eventuali articoli usati come base di partenza per il lavoro svolto ed evidenziare le motivazioni per cui la soluzione fornita può essere ritenuta efficace, trattando in modo succinto le tecniche usate per valutarla.

\medskip

\textbf{Risultati.} Presentiamo in modo chiaro e conciso i risultati raggiunti dalla nostra ricerca, esponendo, dunque, i nuovi metodi, le teorie, i modelli o le implementazioni software prodotte. Se il nostro lavoro è parte di un progetto più grande, è importante specificare chiaramente quale sia stato il nostro contributo a tale progetto.

\medskip

\textbf{Discussione.} Analizziamo se i risultati ottenuti corrispondano agli obiettivi prefissati e se abbiamo risposto alle domande di ricerca. Illustriamo, inoltre, le debolezze e le limitazioni del nostro lavoro, le questioni ancora aperte, le domande emerse e gli eventuali approccio alternativi che avremmo potuto esplorare. Evidenziamo, infine, la rilevanza dei risultati ottenuti, spiegando come questi possano essere applicati nella pratica e come possano contribuire a teorie scientifiche esistenti.

\medskip

\textbf{Conclusioni.} Terminiamo con una o due frasi che riassumono i punti salienti del lavoro svolto e il suo contributo generale. Proponiamo poi dei suggerimenti per potenziali ricerche future che possano estendere o approfondire il nostro studio, come idee non implementate o strategie per superare le limitazioni e le debolezze individuate.

%--------------------------------------------------------------
% 					    Bibliografia
%--------------------------------------------------------------

% La struttura del riassunto della tesi è ispirata a quella suggerita da Männistö et al. per l'abstract di una tesi scientifica e descritta nel documento "Scientific Writing - Guide of the Empirical Software Engineering Research Group of the University of Helsinki", versione 1.7, novembre 2022, https://www.cs.helsinki.fi/group/ese/ScientificWritingGuide.pdf.

% Per il contenuto dei singoli paragrafi ci siamo basati anche sulle indicazioni fornite da:

% - Zobel in "Writing for Computer Science", terza edizione, 2015, https://doi.org/10.1007/978-1-4471-6639-9;

% - Pfandzelter et al. in "Writing a Computer Science Thesis", dicembre 2022, https://github.com/pfandzelter/thesis-tips;

% - Aceto in "How to Write a Paper", slides pubblicate sulla pagina web dedicata ai consigli su come fare ricerca dell’Icelandic Center of Excellence in Theoretical Computer Science, http://icetcs.ru.is/luca/howto-lectures/howtowrite-gssi.pdf, consultata in data 17 novembre 2023;

% - la Tampere University in "Guide to Writing a Thesis in Technical Fields", gennaio 2019, https://content-webapi.tuni.fi/proxy/public/2019-10/tau_thesis_guide_for_technical_fields_2019_version-3-1.pdf;

% - il laboratorio Computer Science 7 (Computer Networks and Communication Systems) della Friedrich-Alexander Universität sulla pagina dedicata alle linee guida per la tesi, https://www.cs7.tf.fau.eu/teaching/student-theses/writing-your-thesis/, consultata in data 16 novembre 2023.